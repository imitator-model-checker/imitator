\documentclass{llncs}

%%%%%%%%%%%%%%%%%%%%%%%%%%%%%%%%%%%%%%%%%%%%%%%%%%%%%%%%%%%%
% PACKAGES
%%%%%%%%%%%%%%%%%%%%%%%%%%%%%%%%%%%%%%%%%%%%%%%%%%%%%%%%%%%%
\usepackage[utf8x]{inputenc}

% SYMBOLES
\usepackage{amsmath, amssymb, url} % amsthm, 


%%%%%%%%%%%%%%%%%%%%%%%%%%%%%%%%%%%%%%%%%%%%%%%%%%%%%%%%%%%%
% TIKZ
%%%%%%%%%%%%%%%%%%%%%%%%%%%%%%%%%%%%%%%%%%%%%%%%%%%%%%%%%%%%
\usepackage{pgf}
\usepackage{tikz}

\graphicspath{{./figures/}}

% Couleurs
\definecolor{turquoise}{rgb}{0 0.41 0.41}
\definecolor{rouge}{rgb}{0.79 0.0 0.1}
\definecolor{vert}{rgb}{0.15 0.4 0.1}
\definecolor{mauve}{rgb}{0.6 0.4 0.8}
\definecolor{violet}{rgb}{0.58 0. 0.41}
\definecolor{orange}{rgb}{0.8 0.4 0.2}
\definecolor{bleu}{rgb}{0.39, 0.58, 0.93}
\definecolor{gris}{rgb}{0.6,0.6,0.6}
\definecolor{grisfonce}{rgb}{0.4, 0.4, 0.4}
\definecolor{grispale}{rgb}{0.9, 0.9, 0.9}
% NOIR ET BLANC
\definecolor{vertfonce}{rgb}{0.0, 0, 0.0}
\definecolor{rougefonce}{rgb}{0, 0.0, 0.0}
\definecolor{rougeclair}{rgb}{0.6, 0.6, 0.6} %{red!50!white}
\definecolor{bleufonce}{rgb}{0.2, 0.2, 0.2} %blue!80!black
\definecolor{bleutresfonce}{rgb}{0.1, 0.1, 0.1} %blue!50!black
% \definecolor{vertfonce}{rgb}{0.0, 0.5, 0.0}
% \definecolor{rougefonce}{rgb}{1, 0.0, 0.0}
% \definecolor{rougeclair}{rgb}{1, 0.5, 0.5} %{red!50!white}
% \definecolor{bleufonce}{rgb}{0, 0, 0.8} %blue!80!black
% \definecolor{bleutresfonce}{rgb}{0, 0, 0.5} %blue!50!black


% Jeu de couleurs pales
% NOIR ET BLANC
\definecolor{cpale1}{rgb}{0.6, 0.6, 0.6}
\definecolor{cpale2}{rgb}{0.9, 0.9, 0.9}
% \definecolor{cpale1}{rgb}{1, 0.6, 0.6}
% \definecolor{cpale2}{rgb}{0.6, 1, 0.6}
\definecolor{cpale3}{rgb}{0.6, 0.6, 1}
\definecolor{cpale4}{rgb}{1, 0.6, 1}
\definecolor{cpale5}{rgb}{1, 1, 0.6}
\definecolor{cpale6}{rgb}{0.6, 1, 1}
\definecolor{cpale7}{rgb}{0.95, 0.65, 0.25}
\definecolor{cpale8}{rgb}{0.75, 0.45, 1}
\definecolor{cpale9}{rgb}{0.5, 1, 0.75}
\definecolor{cpale10}{rgb}{0.8, 0.7, 0.6}
\definecolor{cpale11}{rgb}{0.6, 0.7, 0.8}
\definecolor{cpale12}{rgb}{0.2, 0.5, 0.9}
\definecolor{cpale13}{rgb}{0.5, 0.9, 0.2}
\definecolor{cpale14}{rgb}{0.9, 0.2, 0.5}
\definecolor{cpale15}{rgb}{0.7, 0.7, 0.7}
\definecolor{cpale16}{rgb}{0.8, 0.8, 0.5}



%%%%%%%%%%%%%%%%%%%%%%%%%%%%%%%%%%%%%%%%%%%%%%%%%%%%%%%%%%%%
% CONSTANTES
%%%%%%%%%%%%%%%%%%%%%%%%%%%%%%%%%%%%%%%%%%%%%%%%%%%%%%%%%%%%

%-%-%-%-%-%-%-%-%-%-%-%-%-%-%-%-%-%-%-%-%-%-%-%-%-%-%-%-%-%
% CONSTANTES MATHEMATIQUES
%-%-%-%-%-%-%-%-%-%-%-%-%-%-%-%-%-%-%-%-%-%-%-%-%-%-%-%-%-%
% Booleens
\newcommand{\false}{\mathit{False}}
\newcommand{\true}{\mathit{True}}

% Ensembles
\newcommand{\AP}{\mathit{AP}}
\newcommand{\grandn}{{\mathbb N}}
\newcommand{\grandq}{{\mathbb Q}}
\newcommand{\grandqplus}{{\mathbb Q}_{\geq 0}}
\newcommand{\grandr}{{\mathbb R}}
\newcommand{\grandrplus}{\grandr_{\geq 0}}
\newcommand{\grandz}{{\mathbb Z}}
\newcommand{\setX}{\mathcal{K}_{\Clock}}
\newcommand{\setP}{\mathcal{K}_{\Param}}
\newcommand{\setXP}{\mathcal{K}_{\Clock \cup \Param}}


% Noms
% \newcommand{\tiling}{Tiling} % \mathit{Tiling}
\newcommand{\post}{\mathit{Post}}
\newcommand{\Pre}{\mathit{Pre}}
\newcommand{\TS}{\mathit{TS}}
\newcommand{\words}{\mathit{Words}}

% Operateurs
% \newcommand{\eqdef}{ \overset{\mathit{def}}{=}}


% Symboles
\newcommand{\cro}[1]{\langle #1 \rangle}
\newcommand{\fleche}[1]{\stackrel{#1}{\rightarrow}}
\newcommand{\Fleche}[1]{\stackrel{#1}{\Rightarrow}}
% \newcommand{\prefix}[2]{ |#1|_{#2} }
\newcommand{\steps}[0]{ {\rightarrow} }
\newcommand{\Steps}[0]{ {\Rightarrow} }
\newcommand{\timelaps}[1]{#1^\uparrow}
\newcommand{\urgent}[0]{\twoheadrightarrow}
\newcommand{\wpi}[1]{\mathbin{<}#1\mathbin{>}}

% Redfinition of CSP constructs
\newcommand{\trace}[1]{\langle #1 \rangle}
\newcommand{\fun}{\rightarrow}
\newcommand{\hide}[0]{ \  {\backslash} \, }
\newcommand{\sdef}{\doteq} % \corresponds \hateq
\newcommand{\tun}{\hspace{1cm}} % formerly t1!
\newcommand{\srule}[3]{
\infer[ (#1) ]
     {#3}
     {#2}
     \bigskip
}






% Unites
% \newcommand{\micros}{\mathit{\mu\,s}}
% \newcommand{\millis}{\mathit{m\,s}}
% \newcommand{\nanos}{ns}
% \newcommand{\picos}{ps}


% Variables
\newcommand{\A}{\mathsf{A}} % \mathcal{A}
\newcommand{\activation}{\mathit{Act}} % \mathsf{A}
\newcommand{\counter}{C}
% \newcommand{\enabled}{\mathit{enabled}}
\newcommand{\enev}{\mathit{active}} % Active / enables events
\newcommand{\formule}{\varphi}
\newcommand{\Formule}{\Phi}
\newcommand{\G}{\mathcal{G}}
\newcommand{\LTS}{\mathcal{L}}
\newcommand{\Clock}{X} % set of clocks
\newcommand{\Clocks}{\mathcal{X}} % set of clocks
\newcommand{\clock}{x} % clock
\newcommand{\clockval}{w} % clock valuation
\newcommand{\events}{E} % a possible set of events
\newcommand{\EventsSet}{\Sigma} % all possible events
\newcommand{\instr}{I} % instruction
\newcommand{\Kinit}{K_0} % {\mathit{init}} initial constraint on the parameters
\newcommand{\Machine}{\mathit{CM}} % 2-counter machine
\newcommand{\Param}{U} % set of parameters (P / Y)
\newcommand{\Params}{\mathcal{U}} % set of parameters (P / Y)
\newcommand{\param}{u} % parameter (p / y)
\newcommand{\Processes}{\mathcal{P}}
\newcommand{\ProcessesAct}{\Processes_\mathit{act}}
\newcommand{\PS}{\mathsf{M}} % \mathcal{M} PSTCSP model
\newcommand{\PSex}{\PS_\mathit{ex}}
\newcommand{\py}{\pi} % parameter valuation
\newcommand{\Py}{\Pi} % set of consistent parameter valuations
\newcommand{\runs}{\mathit{Runs}} % set of runs
\newcommand{\sinit}{s_0} % {init} initial set of states
% \newcommand{\Slast}{S_{last}}
\newcommand{\varCl}{\mathit{cl}}
\newcommand{\varIdle}{\mathit{idle}}
\newcommand{\ruleIdle}{\mathit{idle}}
\newcommand{\Val}{\mathcal{V}}
\newcommand{\Var}{\mathit{Var}}
\newcommand{\Vars}{\mathcal{V}ar}
\newcommand{\Vinit}{V_0} % \mathit{init} initial valuation of the variables

% EXAMPLES
\newcommand{\varActive}{\mathit{Active}}
\newcommand{\varCnt}{\mathit{cnt}}
\newcommand{\varExit}{\mathit{exit}}
\newcommand{\varProc}{\mathit{proc}}
\newcommand{\varProtocol}{\mathit{FME}}
\newcommand{\varTurn}{\mathit{turn}}
\newcommand{\varUpdate}{\mathit{update}}

\newcommand{\FischerEpsilon}{\gamma}
\newcommand{\FischerDelta}{\delta}

%-%-%-%-%-%-%-%-%-%-%-%-%-%-%-%-%-%-%-%-%-%-%-%-%-%-%-%-%-%
% ALGORITHMES
%-%-%-%-%-%-%-%-%-%-%-%-%-%-%-%-%-%-%-%-%-%-%-%-%-%-%-%-%-%
% Algorithmes PTA
\newcommand{\explo}{\mathit{reachAll}}
\newcommand{\IM}{\mathit{IM}} % \mathit{IM}


%-%-%-%-%-%-%-%-%-%-%-%-%-%-%-%-%-%-%-%-%-%-%-%-%-%-%-%-%-%
% CONSTANTES DE CHAINES
%-%-%-%-%-%-%-%-%-%-%-%-%-%-%-%-%-%-%-%-%-%-%-%-%-%-%-%-%-%

% Outils
% \newcommand{\apron}{\textsc{Apron}}
% \newcommand{\cplusplus}{C++}
% \newcommand{\gdot}{\textsc{dot}}
% \newcommand{\graphviz}{Graphviz}
\newcommand{\hytech}{{\sc HyTech}}
% \newcommand{\imitator}{\textsc{Imitator}}
\newcommand{\imitatordeux}{\textsc{Imitator}\,II}
% \newcommand{\imperator}{\textsc{ImpRator}}
% \newcommand{\inspeqtor}{\textsc{InSPEQTor}}
% \newcommand{\java}{Java}
% \newcommand{\kronos}{\textsc{Kronos}}
% \newcommand{\nusmv}{NuSMV}
\newcommand{\ocaml}{OCaml}
% \newcommand{\phaver}{PHAVer}
% \newcommand{\plot}{gnuplot}
% \newcommand{\polka}{NewPolka}
% \newcommand{\prism}{\textsc{Prism}}
% \newcommand{\python}{Python}
% \newcommand{\red}{RED}
\newcommand{\romeo}{Rom\'eo}
% \newcommand{\smv}{SMV}
\newcommand{\spark}{\textsc{Spark}}
\newcommand{\spin}{Spin}
% \newcommand{\tina}{TINA}
\newcommand{\trex}{\textsc{TReX}}
\newcommand{\uppaal}{\textsc{Uppaal}}
% \newcommand{\vhdlta}{\textsc{Vhdl2Ta}}

% PSTCSP constructs
% \newcommand{\CSPdeadline}{\mathit{deadline}}
% \newcommand{\CSPinterrupt}{\mathit{interrupt}}
% \newcommand{\CSPprogram}{\mathit{program}}
% \newcommand{\CSPskip}{\mathit{Skip}}
% \newcommand{\CSPstop}{\mathit{Stop}}
% \newcommand{\CSPtimeout}{\mathit{timeout}}
% \newcommand{\CSPwait}{\mathit{Wait}}
% \newcommand{\CSPwithin}{\mathit{within}}
% \newcommand{\CSPdeadline}{ {ddl} }
% \newcommand{\CSPinterrupt}{{inpt}}
% \newcommand{\CSPprogram}{{prg}}
% \newcommand{\CSPskip}{{Skip}}
% \newcommand{\CSPstop}{{Stop}}
% \newcommand{\CSPtimeout}{ {tout} }
% \newcommand{\CSPwait}{{Wait}}
% \newcommand{\CSPwithin}{{win}}
\newcommand{\CSPdeadline}{\mathtt{deadline} }
\newcommand{\CSPelse}{\mathtt{else}}
\newcommand{\CSPif}{\mathtt{if}}
\newcommand{\CSPthen}{\mathtt{then}}
\newcommand{\CSPinterrupt}{\mathtt{interrupt}}
\newcommand{\CSPprogram}{\mathit{program}}
\newcommand{\CSPskip}{\mathtt{Skip}}
\newcommand{\CSPstop}{\mathtt{Stop}}
\newcommand{\CSPtimeout}{\mathtt{timeout} }
\newcommand{\CSPwait}{\mathtt{Wait}}
\newcommand{\CSPwithin}{\mathtt{within}}


%%%%%%%%%%%%%%%%%%%%%%%%%%%%%%%%%%%%%%%%%%%%%%%%%%%%%%%%%%%%
% FORMATING
%%%%%%%%%%%%%%%%%%%%%%%%%%%%%%%%%%%%%%%%%%%%%%%%%%%%%%%%%%%%

% CODE INTEGRE AU TEXTE
\newcommand{\code}[1]{\textbf{\texttt{#1}}}

% COMMENTAIRES DANS UN PARAGRAPHE
% \newcommand{\commentaire}[1]{\textcolor{red}{\textbf{$\Leftarrow$  #1 $\Rightarrow$}}}
\newcommand{\commentaire}[1]{}

% REDEFINITIONS DES PARAGRAPHES
\newcommand{\paragraphe}[1]{\paragraph{#1.}}
% \newcommand{\paragraphe}[1]{\subsubsection{#1.}}

\renewcommand{\baselinestretch}{0.98}
\addtolength{\textheight}{0.2cm}

% Title
\title{Parameter Synthesis for Hierarchical Concurrent Real-Time Systems Using TOOL}
\author{\'Etienne Andr\'e$^1$, Yang Liu$^2$, Jun Sun$^3$ and Jin-Song Dong$^4$}
\institute{$^1$LIPN, CNRS UMR 7030, Université Paris 13, France \\
	$^2$ Temasek Laboratories, National University of Singapore \\
	$^3$ Singapore University of Technology and Design \\
	$^4$ School of Computing, National University of Singapore
}

\begin{document}

\maketitle

\begin{abstract}
	
\end{abstract}

\keywords{Hybrid automata, Verification, Parameter synthesis, Robustness}

\commentaire{Version avec commentaires}


%%%%%%%%%%%%%%%%%%%%%%%%%%%%%%%%%%%%%%%%%%%%%%%%%%%%%%%%%%%%
%%%%%%%%%%%%%%%%%%%%%%%%%%%%%%%%%%%%%%%%%%%%%%%%%%%%%%%%%%%%
\section{Motivation}
%%%%%%%%%%%%%%%%%%%%%%%%%%%%%%%%%%%%%%%%%%%%%%%%%%%%%%%%%%%%
%%%%%%%%%%%%%%%%%%%%%%%%%%%%%%%%%%%%%%%%%%%%%%%%%%%%%%%%%%%%


The specification and verification of real-time systems, involving complex data structures and timing delays, are notoriously difficult problems.
The correctness of such real-time systems usually depends on the values of these timing delays.
One can check the correctness for one particular value for each delay, using classical techniques of timed model checking, but this does not guarantee the correctness for other values.
Actually, checking the correctness for all possible delays, even in a bounded interval, would require an infinite number of calls to the model checker, because those delays can have real (or rational) values.
It is therefore interesting to reason \emph{parametrically}, by considering that these delays are unknown constants, or \emph{parameters}, and try to synthesize a constraint (a conjunction of linear inequalities) on these parameters guaranteeing a correct behavior.

%-%-%-%-%-%-%-%-%-%-%-%-%-%-%-%-%-%-%-%-%-%-%-%-%-%-%-%-%-%
\paragraphe{Motivation}
%-%-%-%-%-%-%-%-%-%-%-%-%-%-%-%-%-%-%-%-%-%-%-%-%-%-%-%-%-%
We are interested here in the \emph{good parameters problem} for real-time systems:
``find a set of parameter valuations for which the system is correct''.
This problem stands between verification and control, in the sense that we actually change (the timing part of) the system in order to guarantee some property.
% This notion of correctness can refer to any kind of properties.
%
Furthermore, we aim at defining a formalism that is intuitive, powerful (with use of external variables, structures and user defined functions), and allowing efficient parameter synthesis and verification.


\cite{ALSD12}

\cite{AD94}

%%%%%%%%%%%%%%%%%%%%%%%%%%%%%%%%%%%%%%%%%%%%%%%%%%%%%%%%%%%%
%%%%%%%%%%%%%%%%%%%%%%%%%%%%%%%%%%%%%%%%%%%%%%%%%%%%%%%%%%%%
\section{Implementation and Features}
%%%%%%%%%%%%%%%%%%%%%%%%%%%%%%%%%%%%%%%%%%%%%%%%%%%%%%%%%%%%
%%%%%%%%%%%%%%%%%%%%%%%%%%%%%%%%%%%%%%%%%%%%%%%%%%%%%%%%%%%%

% %-%-%-%-%-%-%-%-%-%-%-%-%-%-%-%-%-%-%-%-%-%-%-%-%-%-%-%-%-%-%
% \begin{figure}[ht!]
% 	% STYLES
% 	\tikzstyle{etiquette} = [draw=none, color=black]
% 	
% % \tikzstyle{boite}=[text width=8em, text centered, minimum height=2.5em, rounded corners, very thick]
% % \tikzstyle{input}=[boite, draw=green!20!gris, top color=green!50, bottom color=green!10!white]
% % \tikzstyle{output}=[boite, draw=red!20!gris, top color=red!50, bottom color=red!5!white]
% % \tikzstyle{imitator} = [boite, draw=blue!20!gris, text width=10em, minimum height=6em]
% 
% 	
% 	\tikzstyle{boite}=[rectangle, draw=black, rounded corners, thick, draw=blue!40!black, top color=blue!20, bottom color=blue!5!white]
% % 	\tikzstyle{imitator} = [boite]
% % 	\tikzstyle{apron} = [boite, draw=yellow!20!gris, top color=yellow!70, bottom color=yellow!10!white]
% % 	\tikzstyle{polka} = [apron]
% % 	\tikzstyle{ppl} = [apron]
% 	\tikzstyle{dot} = [boite, draw=purple!20!gris, top color=purple!60, bottom color=purple!10!white]
% 
% 	\tikzstyle{fleche} = [->, draw=black, semithick]
% {
% 
% \centering
% 
% \begin{tikzpicture}[scale=0.6,  =>stealth']
% 
% 	% Boites
% 	\draw[boite] (-5, 5.5) rectangle (-2, 6.5);
% 	\node [etiquette] at (-3.5, 6) {Model};
% 	
% 	\draw[boite] (-5, 3.8) rectangle (-2, 5);
% 	\node [etiquette] at (-3.5, 4.4) {\begin{tabular}{c}Reference\\valuation\end{tabular}};
% 	
% 	\draw[boite] (0, 4) rectangle (6, 6.5);
% 	\node [etiquette] at (3, 5.25) {\large \hymitator{}};
% 	
% 	\draw[boite] (1, 2.5) rectangle (5, 3.5);
% 	\node [etiquette] at (3, 3) {PPL};
% 	
% 	\draw[boite] (8, 5.5) rectangle (12, 6.5);
% 	\node [etiquette] at (10, 6) {Constraint};
% 	
% 	\draw[boite] (8, 4) rectangle (12, 5);
% 	\node [etiquette] at (10, 4.5) {Trace set};
% 
% 	% Fleches
% 	\path[fleche] (-2, 6) --++ (2, 0);
% 	\path[fleche] (-2, 4.4) --++ (2, 0);
% 	\path[fleche] (3.5, 4) --++ (0, -.5);
% 	\path[fleche] (2.5, 3.5) --++ (0, .5);
% 	\path[fleche] (6, 6) --++ (2, 0);
% 	\path[fleche] (6, 4.5) --++ (2, 0);
% 
% \end{tikzpicture}
% 
% }
% 
% \caption{Architecture of \hymitator{}}
% \label{fig:structure}
% \end{figure}
% %-%-%-%-%-%-%-%-%-%-%-%-%-%-%-%-%-%-%-%-%-%-%-%-%-%-%-%-%-%-%

TOOL (with sources, binaries and case studies) is available on its Web page: \url{http://www-lipn.univ-paris13.fr/~andre/software/hymitator/}.



%%%%%%%%%%%%%%%%%%%%%%%%%%%%%%%%%%%%%%%%%%%%%%%%%%%%%%%%%%%%
%%%%%%%%%%%%%%%%%%%%%%%%%%%%%%%%%%%%%%%%%%%%%%%%%%%%%%%%%%%%
\section{Applications}
%%%%%%%%%%%%%%%%%%%%%%%%%%%%%%%%%%%%%%%%%%%%%%%%%%%%%%%%%%%%
%%%%%%%%%%%%%%%%%%%%%%%%%%%%%%%%%%%%%%%%%%%%%%%%%%%%%%%%%%%%


%%%%%%%%%%%%%%%%%%%%%%%%%%%%%%%%%%%%%%%%%%%%%%%%%%%%%%%%%%%%
%%%%%%%%%%%%%%%%%%%%%%%%%%%%%%%%%%%%%%%%%%%%%%%%%%%%%%%%%%%%
\section{Related Work}
%%%%%%%%%%%%%%%%%%%%%%%%%%%%%%%%%%%%%%%%%%%%%%%%%%%%%%%%%%%%
%%%%%%%%%%%%%%%%%%%%%%%%%%%%%%%%%%%%%%%%%%%%%%%%%%%%%%%%%%%%


%%%%%%%%%%%%%%%%%%%%%%%%%%%%%%%%%%%%%%%%%%%%%%%%%%%%%%%%%%%%
%%%%%%%%%%%%%%%%%%%%%%%%%%%%%%%%%%%%%%%%%%%%%%%%%%%%%%%%%%%%
%%%%%%%%%%%%%%%%%%%%%%%%%%%%%%%%%%%%%%%%%%%%%%%%%%%%%%%%%%%%
\bibliographystyle{abbrv} % alpha plain
\bibliography{biblioPSTCSP}
%%%%%%%%%%%%%%%%%%%%%%%%%%%%%%%%%%%%%%%%%%%%%%%%%%%%%%%%%%%%
%%%%%%%%%%%%%%%%%%%%%%%%%%%%%%%%%%%%%%%%%%%%%%%%%%%%%%%%%%%%
%%%%%%%%%%%%%%%%%%%%%%%%%%%%%%%%%%%%%%%%%%%%%%%%%%%%%%%%%%%%


\end{document}          
